\documentclass{article}

%----------------------------------------
% Packages
%----------------------------------------
\usepackage[left=1in, right=1in, top=1in, bottom=1in]{geometry}
\usepackage{graphicx}
\usepackage{amsmath,amsbsy,amssymb,amsfonts,amsthm}
\usepackage{nicefrac}
\usepackage{mathtools}
\usepackage{color}
\usepackage{xspace} % Correct macro spacing
\usepackage[numbers]{natbib} % For citations
\usepackage{times}
\usepackage{textcomp}
\usepackage{graphicx,subfigure}
%\usepackage[small,bf]{caption}
\usepackage{algorithm,algorithmic} 
\usepackage{hyperref}
\usepackage{comment}

\usepackage{mathrsfs}

\usepackage{xcolor}
\usepackage{shadethm}

\usepackage{fancyhdr}
\usepackage{wrapfig}
\usepackage{tikz}
\usetikzlibrary{arrows}
\usetikzlibrary{positioning}
\usetikzlibrary{decorations.pathreplacing}
\pagestyle{fancy}
\lhead{This is my name}
\rhead{this is page \thepage}

\usepackage{fancyhdr}
\pagestyle{fancy}
\lhead{IFT 6085 - Theoretical principles for deep learning}
\rhead{Lecture 2: January 10, 2019}


\newshadetheorem{thm}{Theorem}
\newshadetheorem{defn}[thm]{Definition}
\newshadetheorem{assm}[thm]{Assumption}
\newshadetheorem{prop}[thm]{Property}
\newshadetheorem{eg}[thm]{Example}
\newshadetheorem{lemma}[thm]{Lemma}

\definecolor{shadethmcolor}{HTML}{F0F0F0}
%\definecolor{shadethmcolor}{HTML}{EDEDED}
%\definecolor{shadethmcolor}{HTML}{EDF8FF}
%\definecolor{shaderulecolor}{HTML}{EDF8FF}
%\definecolor{shaderulecolor}{HTML}{45CFFF}
\setlength{\shadeboxrule}{.4pt}


\setlength\parindent{0pt}

% Packages hyperref and algorithmic misbehave sometimes.  We can fix
% this with the following command.
\newcommand{\theHalgorithm}{\arabic{algorithm}}

%----------------------------------------
% Standard macros
%----------------------------------------


%----------------------------------------
% Project-specific macros
%----------------------------------------

%----------------------------------------
% Header
%----------------------------------------
\title{IFT 6085 - Lecture 9 \\ Stability, Generalization and the Applications of Stability}
\date{}

%----------------------------------------
% Document
%----------------------------------------
\begin{document} 

%----------------------------------------
% Abstract
%----------------------------------------
\maketitle

\vspace{-0.5in}
\begin{center}
This version of the notes has not yet been thoroughly checked.
Please report any bugs to the scribes or instructor.
\end{center}
\vspace{0.2in}

\textbf{Scribes}\hfill
\textbf{Instructor:}  Ioannis Mitliagkas\\
\textbf{Winter 2021:} Hattie Zhou\\
\textbf{Winter 2020:} Zarour Mahdi\\
\textbf{Winter 2019:} Nishanth V Anand, Parviz Haggi,
Bhargav Kanuparthi, Jonathan Pilault\\
\textbf{Winter 2018:} Isabela Albuquerque, Nithin Vasisth, Amy Zhang, William Fedus


%----------------------------------------
% Body
%----------------------------------------

\newcommand{\infgc}{\inf_{g \in \mathcal{C}}}
\newcommand{\supgc}{\sup_{g \in \mathcal{C}}}

\newcommand{\Prob}{\mathbb{P}}
\newcommand{\E}{\mathbb{E}}
\newcommand{\reals}{\mathbb{R}}

\section{Summary of the previous lecture}
Previously we discussed the idea behind a few bounds in Statistical Learning Theory, the more advanced one being PAC-based Learning Theory. There we saw that we have to commit to a prior distribution on a hypothesis (P) and choose the posterior hypothesis (Q) after seeing the data. This is a powerful method as different choices of prior and posterior hypotheses can be made, each resulting in a new bound without us touching the algorithm. 
In a coming lecture we will discuss a concrete example of a PAC-based bound for Neural Networks. There, the discussion will also include the notion of stability-based bounds, which is the subject of this lecture.\\


The algorithm-agnostic bounds take into account the complexity of the hypothesis class of functions $\mathcal{H}$, without involving the learning algorithm or the actual distribution of the data. More precisely, in order to use Hoeffding's inequality, we assumed that the data points were i.i.d.. Apart from this assumption, however, no other information about the data distribution was used. \\


Although we will not delve into the subject of distribution-agnostic bounds, in this lecture we will introduce the first class of bounds that takes into account the learning algorithm. The analysis is largely dependent on the notion of stability which, simply put, says that a change in data distribution does not change the predictions. 



\section{PAC Learning}
The setting of the Probably Approximately Correct (PAC) learning involves the same definitions as before, but we introduce a slight change in notation that will help us in our analysis. We now denote each data pair as $z_i=(x_i,y_i)$.

\begin{defn}[Training set]
The training set consists of a set of values $z_i=(x_i,y_i)$, where $x_i$ represents a feature vector and $y_i$ the label of the $i$-th sample. Furthermore, $z_i$ are assumed to be i.i.d. and sampled from an unknown data distribution $\mathcal{D}$. We denote the sampled dataset as $S$. 
\[
	S = \{z_1, z_2, \dots, z_n\}
\]
\end{defn}
 Next we define the loss function slightly differently. 
\begin{defn}[Loss Function] The loss function $\ell(h(x),y)$ is defined as a function that takes two labels and produces a value between $0$ and some constant $M$. 
\[
	\ell: \mathcal{Y} \times \mathcal{Y} \longrightarrow [0,M].
\]
Equivalently, defining $\mathcal{Z}\overset{\Delta}{=}\mathcal{X}\times \mathcal{Y}$,  the loss function $\ell (h,z)$  can be defined as 
\[
	\ell: \mathcal{H} \times \mathcal{Z} \longrightarrow [0,M].
\]
\end{defn}
Notice that, unlike its previous definition, the loss function is now assumed to be bounded by some constant $M$ instead of 1. 
Note also that $\mathcal{Z}$ is the space of all tuples $(x_i,y_i) $ and that the previous loss function $\ell(h(x),y)$ is now denoted as $\ell(h,z)$.


\section{Stability}
We start this section by introducing two new notions, the first of which is the notion of  perturbed datasets. This is a step-stone in introducing a new class of bounds that unlike the   \textit{algorithm-agnostic} bounds, are dependent on the algorithm. 
\begin{defn}[A Perturbed dataset] Given a dataset $S$ (i.i.d.), a perturbed dataset $S^{i, z}$ is defined as
\[
	S^{i,z} = \{z_1, z_2, \dots, z_{i-1}, z, z_{i+1}, \dots, z_{n} \}.
\]
\end{defn}
According to this definition, a perturbed dataset $S^{i,z}$ is defined by a set whose $i$-th element is replaced by an arbitrary sample $z$. 
We will see that some learning algorithms produce a hypothesis that makes approximately the same predictions whether the algorithm is trained on the original dataset or the perturbed dataset.
\begin{defn}[Algorithm]
A learning algorithm $\mathcal{A}$ is defined as the following mapping
\[
\mathcal{A}: (\mathcal{X} \times \mathcal{Y})^n \rightarrow \mathcal{H}.
\]
\end{defn}
It is clear from this definition that an algorithm $\mathcal{A}$ applied on a dataset $S$, produces a hypothesis $h_{S}$, i.e. $h_S = \mathcal{A}(S)$.


\begin{defn}[Uniform stability] An algorithm $\mathcal{A}$ is $\beta$-uniformly stable with respect to the loss function $\ell$ if
\[\forall (S,z) \in \mathcal{Z}^{n+1}\;\; \text{and} \;\;\forall i \in \{1, 2, \dots, n\}:\;\;\;
	\sup_{z' \in \mathcal{Z}} | \ell(h_S,z') - \ell(h_{S^{i, z}},z' ) | \le \beta.
\]
\end{defn}
This definition measures stability based on how much the predictions or the losses on the predictions change when we train using the perturbed dataset. Notice the two hypotheses: $h_S$ comes from using the unperturbed dataset, and $h_{S^{i,z}}$ comes from using the perturbed dataset. This definition holds for any dataset and there is no assumption on which particular distribution $S$ comes from. The notion of a $\beta$-uniformly stable algorithm is reminiscent of the familiar notion of Lipschitz property on the loss function. Intuitively, an algorithm with this property can be understood as one that produces a hypothesis such that the loss function $\ell$ is not drastically affected by perturbing the dataset in this manner.\\ 
 
 %It is important to emphasize that this is a uniform bound because it is valid for all datasets $S$ and indexes $i$. \\ %We usually do not need such a strong statement, we can restrict the values of $S$ and $z$ to be within the region of the data.

%*The concept of robustness is not the same




 
\begin{defn}[Defect]\label{defect}
\[
	D[h_S] = R[h_{S}] - \hat{R}_{S}[h_S].
\]
\end{defn}
Defect $D[h_S]$ for a hypothesis $h_S$ derived from an algorithm after seeing the dataset $S$ is defined as  the difference between the population risk and the empirical risk. \\

While it is true that 
for an arbitrary hypothesis $h \in \mathcal{H}$,\;
	$\mathbb{E}[{D}[h]]=\mathbb{E}[R[h] - \hat{R}_S[h]]=0$,
this is not the case for $D[h_S]$, i.e. we will generally have   
\[
	\mathbb{E}[D[h_S]] \neq 0.
\]
This is due to second term $\hat{R}_{S}[h_S]$ which evaluates the empirical risk on the same  dataset that is also used to extract the hypothesis. As seen in previous lectures, we lose the i.i.d. assumption in this case.

\begin{align*}
    \mathbb{E}_S[\hat{R}_{S}[h_S]]
    &=\mathbb{E}_S[\frac{1}{n}\sum_{i=1}^n \ell(h_S(x_{i}),y_{i})]\\
    &=\frac{1}{n}\sum_{i=1}^n\mathbb{E}_S[ \ell(h_S(x_{i}),y_{i})]\\
    &\neq \mathbb{E}_S[\mathbb{E}_{z \sim \mathcal{D}} \left[\ell(h_S(x),y)\right]]\overset{\Delta}{=}\mathbb{E}_S[{R}[h_S]]
\end{align*}
In the second line of the above equation, the expectation value is evaluated based on the dataset $S$ and the hypothesis extracted from it.  
The expectation value of the population risk on the learned hypothesis (third line) is evaluated on a random variable $z$, independent from $S$ and drawn from the distribution that the dataset is assumed to come from. 
The two entities are generally completely different simply because we need the independence assumption for the population risk.\\ 

A meaningful question to ask here is whether the expectation value of the defect, which we showed is generally non-zero, can be bounded. In what follows, we will show that the answer is yes under certain conditions.  
  	

\begin{thm}[Bounding the expectation value of the defect]\label{defec_bound}
If $\mathcal{A}$ is a $\beta$-uniformly stable algorithm, then 
\[
	-\beta \le \mathbb{E}_{S}[D[h_S]] \le \beta.
\]
\end{thm}
\begin{proof}
We prove this for one side of the inequality:  
	 $-\mathbb{E}_{S}[D[h_S]] \le \beta$. The proof for the other side can be done using the same arguments.

\begin{align*}
    -\mathbb{E}_S[D[h_S]]
    &=\mathbb{E}_S\Big[
    \hat R_S[h_S]-R[h_S]\Big]\\
    &=\mathbb{E}_S
    \Big[\frac{1}{n}\sum_{i=1}^n \ell(h_S,z_{i})-\mathbb{E}_z\ell(h_S,z)\Big]\\
    &=\mathbb{E}_{S,z}
    \Big[\frac{1}{n}\sum_{i=1}^n \Big[\ell(h_S,z_{i})-\ell(h_S,z)\Big]\Big]\\
    &=\mathbb{E}_{S,z}
    \Big[\frac{1}{n}\sum_{i=1}^n \Big[\ell(h_{S^{i,z}},z)-\ell(h_S,z)\Big]\Big]\\
    &\leq \mathbb{E}_{S,z}
    \Big[\frac{1}{n}\sum_{i=1}^n\beta\Big]=\beta
\end{align*}	 
\end{proof}
In the second step we inserted the population risk as defined above: $R[h_S]=\mathbb{E}_z\ell(h_S,z)$. In the third step, we used Fubini's theorem which allows us to change the order of the two expectations $\mathbb{E}_{S}$ and $\mathbb{E}_{z}$ as $S$ and $z$ are independent and bounded (due to the fact that the loss function is bounded by $M$). In the fourth step, we renamed a variable (see below), and in the last step we arrived at the upper bound by using the definition of $\beta$-stability. \\

\textbf{Note concerning the $4^{th}$ line of the proof:} \\ 
Notice that getting a hypothesis $h_S$ from dataset $\{z_1, z_2,..., z_n\}$ and evaluating the loss on $z_1$ is equivalent to getting a hypothesis $h_{S^{1,z}}$ from dataset $\{z, z_2,..., z_n\}$ and evaluating the loss on $z$, because each $z_i$ is sampled i.i.d. from $\mathcal{D}$. We can then generalize this argument for $i$ in $\{1, 2,..., n\}$ and rewrite $\ell(h_S,z_{i})$ as $\ell(h_{S^{i,z}},z)$.\\

% - Getting a hypothesis from $(z_1, z_2,..., z_n)$ which is $h_S$ and computing the loss in $z_1$ for example.\\
% - And Getting a hypothesis from $(z, z_2,..., z_n)$ which is $h_{S^{1,z}}$ and computing the loss in $z$ is the same thing. \\
% And we generalize that for $i$ in $\{1, 2,..., n\}$\\

In conclusion we have proven the following relationship between the expectation values of the empirical and population risks: 
\begin{prop}[The relationship between the empirical and the population risk]
\[
	\mathbb{E}_{S}[R[h_{S}]] \le \mathbb{E}_{S}[\hat{R}_{S}[h_S]] + \beta.
\]
\end{prop}








 
 %It is important to emphasize that this is a uniform bound because it is valid for all datasets $S$ and indexes $i$. \\ %We usually do not need such a strong statement, we can restrict the values of $S$ and $z$ to be within the region of the data.

%*The concept of robustness is not the same







Note that this is a bound on the  \textit{expectation} value of the population risk. This means that even if the expectation values of $R[h_S]$ and $\hat{R}_S[h_S]$ are close, this bound does not necessarily hold for all possible $h_S$. 
In what follows, we will demonstrate that for a $\beta$-uniformly stable algorithm, the population risk  $R[h_S]$ can be bounded above by the empirical risk $\hat{R}_S[h_S]$ plus certain other quantities. To do so we will first introduce McDiramid's inequality, a well known concentration inequality.

\begin{thm}[McDiarmid's inequality]
Let $V_1, V_2, V_3, \dots , V_n \in \mathcal{V}$ be independent random variables, and $v_1, v_2,v_3,\dots,v_n$ denote observed instances of the random variables (not random). If a function $f: \mathcal{V}^n \to \mathbb{R}$ has the property that $\forall i \in \{1, 2, \dots, n\}$, 
\[
	\sup_{v_1, v_2, \dots, v_n, v_i{'}}{\biggl|f(v_1, v_2, \dots, v_n) - f(v_1, \dots,v_{i-1}, v_i^{'}, v_{i+1}, \dots v_n)\biggr|} \leq c_i
\]
then
\[
    \mathbb{P}\biggl(\bigl|f(V_1, V_2, \dots , V_n) - \mathbb{E}f(V_1, V_2, \dots , V_n)\bigr|>\epsilon\biggr) \leq 2 \exp{\frac{-2 \epsilon^2}{\sum_{i}{c_{i}^2}}}
\]
\end{thm}

\begin{proof}
See Appendix D.2 of \cite{mohri2012foundations}.
\end{proof}

McDiarmid's inequality gives us concentration over functions of random variables. This bound is useful because if we prove that an algorithm is $\beta$-stable, then we can assign $f$ to be the defect and achieve a bound on a specific hypothesis.

\begin{thm}[Bound for the defect]\label{def_bound} Let "A" be a $\beta$-uniformly stable learning algorithm with respect to a loss function $\ell: \mathcal{Y} \times \mathcal{Y} \rightarrow [0, M]$. The absolute difference of the defect calculated on a dataset $S$ and on a perturbed version of this dataset $S^{i, z}$ is bounded by
\[
\biggl|D[h_S] - D[h_{S^{i,z}}]\biggr| \leq 2\beta + \frac{M}{n}.
\]
\end{thm}
This theorem gives us a bound on the gap between the actual dataset and the perturbed dataset. This is used as a stepping stone in the final theorem. As we will see below, the theorem tells us that the population risk and the empirical risk are close to each other for a given hypothesis. This theorem is an example of the use case of McDiramid's inequality.
\begin{proof}
Let us expand the following quantity using their definition: 
\begin{equation}\label{def_risk}
|D[h_S] - D[h_{S^{i,z}}]| = |R[h_S] - \hat{R}_S[h_S] - R[h_{S^{i,z}}] + \hat{R}_{S^{i,z}}[h_{S^{i,z}}]|
\end{equation}
Using the triangle inequality:
\begin{equation}
|D[h_S] - D[h_{S^{i,z}}]| \leq |R[h_S] - R[h_{S^{i,z}}]| + |\hat{R}_{S^{i,z}}[h_{S^{i,z}}] - \hat{R}_S[h_S]| \label{PartAB}
\end{equation}
Now, we use the $\beta$-uniform stability of algorithm $\mathcal{A}$ with respect to the loss function $\ell$ to find a bound for $|R[h_S] - R[h_{S^{i,z}}]|$:
\begin{equation}
\begin{split}
\bigl|R[h_S] - R[h_{S^{i,z}}]\bigr| &= \bigl| \mathbb{E}_{z' \sim D} [\ell(h_S, z')] - \mathbb{E}_{z' \sim D} [\ell(h_{S^{i,z}}, z')] \bigr| \\ 
&= \bigl|\mathbb{E}_{z' \sim D} [\ell(h_S, z') - \ell(h_{S^{i,z}}, z'))]\bigr| \\
& \leq \beta \label{partA}
\end{split}
\end{equation}
We can find a bound for $|\hat{R}_{S^{i,z}}[h_{S^{i,z}}] - \hat{R}_S[h_S]|$ by expanding the quantities using their definition (notice that the perturbed dataset is just the non-perturbed dataset with $z_i$ subtracted and $z$ added).
\begin{equation}
\begin{split}
\bigl|\hat{R}_{S^{i,z}}[h_{S^{i,z}}] - \hat{R}_S[h_S]\bigr| &= \biggr|\frac{1}{n}\sum_{j=1}^{n}\ell(h_{S^{i,z}}, z_j) - \frac{1}{n}\ell(h_{S^{i,z}}, z_i) + \frac{1}{n}\ell(h_{S^{i,z}}, z) - \frac{1}{n}\sum_{j=1}^{n}\ell(h_S, z_j)\biggl|\\
&\leq \frac{1}{n} \biggl|\ell(h_{S^{i,z}}, z) - \ell(h_{S^{i,z}}, z_i)\biggr| + \frac{1}{n}\sum_j\biggl|\ell(h_{S^{i,z}}, z_j) -  \ell(h_S, z_j)\biggr| \\
&\leq \frac{M}{n} + \beta \label{partB}
\end{split}
\end{equation}
We could do this last step because we know that $\biggl|\ell(h_{S^{i,z}}, z) - \ell(h_{S^{i,z}}, z_i)\biggr|$ is bounded by $M$ from \textbf{Definition 2}.  \\
We now plug in the results obtained from Eq \ref{partA} and Eq \ref{partB} in Eq \ref{PartAB} to get the proof.
\begin{equation}\label{proof1}
\begin{split}
|D[h_S] - D[h_{S^{i,z}}]| & \leq 2\beta + \frac{M}{n}
\end{split}
\end{equation}
\end{proof}

\begin{thm}[Bound for the population risk of a $\beta$-uniformly stable algorithm]\label{risk_bound}
Consider a $\beta$-uniformly stable algorithm $\mathcal{A}$ with respect to a loss function $\ell: \mathcal{Y} \times \mathcal{Y} \rightarrow [0, M]$ and a hypothesis $h_S$ with $|S| = n$. The following bound holds with probability $1-\delta$: 
\[
R[h_S] \leq \hat{R}_S[h_S] + \beta + \left(n\beta + \frac{M}{2} \right) \sqrt{\frac{2\log\frac{2}{\delta}}{n}}.
\]
\end{thm}
Note that:
\begin{itemize}
    \item $\hat{R}_S[h_S]$ is empirical risk
    \item $\beta$ is from theorem \ref{defec_bound}
    \item $\left(n\beta + \frac{M}{2} \right) \sqrt{\frac{2\log\frac{2}{\delta}}{n}}$ is a concentration inequality (McDiramid's)
\end{itemize}
\begin{proof}
Using theorem \ref{def_bound}, we state McDiarmid's inequality for $D[h_S]$ and then use this result to find a high probability bound for $D[h_S]$:
\begin{equation}
\sup_{S, i, z} |D[h_S] - D[h_{S^{i, z}}]| \leq 2\beta + \frac{M}{n}
\end{equation}
then 
\begin{equation}
\begin{split}
P(|D[h_S] - \mathbb{E}[D[h_S]]| > \epsilon) & \leq 2\exp \left(\frac{-2\epsilon^2}{\sum_{i=1}^{n}  \left( 2\beta + \frac{M}{n} \right)^2 } \right), \\
& = 2\exp \left(\frac{-2n\epsilon^2}{\left( 2n\beta + M \right)^2 } \right), \\
& = 2\exp \left(\frac{-2n\epsilon^2}{4\left(n\beta + \frac{M}{2} \right)^2 } \right), \\
& = 2\exp \left(\frac{-n\epsilon^2}{2\left(n\beta + \frac{M}{2} \right)^2 } \right).
\end{split}
\end{equation}

Denoting $\delta = 2\exp \left(\frac{-n\epsilon^2}{2\left(n\beta + \frac{M}{2} \right)^2 } \right)$ and solving this equation for $\epsilon$, we obtain:

\begin{equation}\label{epsilon}
\begin{split}
\delta = 2\exp \left(\frac{-n\epsilon^2}{2\left(n\beta + \frac{M}{2} \right)^2 } \right) & \Rightarrow n\epsilon^2 = 2\log\frac{2}{\delta}\left(n\beta + \frac{M}{2} \right)^2 \\
& \Rightarrow \epsilon = \left(n\beta + \frac{M}{2} \right) \sqrt{\frac{2\log\frac{2}{\delta}}{n}}.
\end{split}
\end{equation}
Thus, with probability $1-\delta$ 
\begin{equation*}
\begin{split}
|D[h_S] - \mathbb{E}[D[h_S]]| &\leq \epsilon \\
D[h_S] &\leq \mathbb{E}[D[h_S]] + \epsilon \\
D[h_S] &\leq \beta + \epsilon.
\end{split}
\end{equation*}
Replacing $\epsilon$ by the result previously obtained in Eq. \ref{epsilon}
\begin{equation}
D[h_S] \leq \beta + \left(n\beta + \frac{M}{2} \right) \sqrt{\frac{2\log\frac{2}{\delta}}{n}}.
\end{equation}
We finally get the desired result,
\begin{equation}
R[h_S] \leq \hat{R}_S[h_S] + \beta + \left(n\beta + \frac{M}{2} \right) \sqrt{\frac{2\log\frac{2}{\delta}}{n}}.
\end{equation}
\end{proof}

\textbf{Notes:}\\
% - Notice that as $n$ goes up, our bound becomes less tight (even vacuous) which is not what we want. So we are not satisfied with our current results.\\
The term $(\beta n + \frac{M}{2})\sqrt{\frac{2\ln{2/\delta}}{n}}$ is $O(\beta\sqrt{n})$. Informally, we say an algorithm is stable if $\beta=O(\frac{1}{n})$. If $\beta$ is $O(\frac{1}{\sqrt{n}})$ for example, this term becomes $O(1)$ and we can no longer show decrease in generalization gap with increase in $n$.\\


The following illustration represents a summary of the bounds stated in Theorem \ref{defec_bound} and \ref{risk_bound} (in terms of the defect):
\begin{figure}[H]\centering
  \begin{tikzpicture}[scale=3]
    \draw[->] (-1,0) -- (1,0);
    \draw[dashed] (-0.7,0) -- (-0.7,0.25);
    \draw[dashed] (0.7,0) -- (0.7,0.25);
	\draw[-] (-0.3,-0.03) -- (-0.3,0.03);
    \draw[] (-0.3,-0.03) node[below=-0.1cm] {\small$-\beta$};
    \draw[-] (0.3,-0.03) -- (0.3,0.03);
    \draw[] (0.3,-0.03) node[below=-0.1cm] {\small$\beta$};
    \draw[-] (-0.7,-0.03) -- (-0.7,0.03);
    \draw[] (-0.7,-0.03) node[below=-0.1cm] {\small$-\beta-\epsilon$};
    \draw[-] (0.7,-0.03) -- (0.7,0.03);
    \draw[] (0.7,-0.03) node[below=-0.1cm] {\small$\beta+\epsilon$};
    \draw [decorate, decoration={brace, amplitude=10pt}] (-0.7,0.25) -- (0.7,0.25) node [black, midway, yshift=0.6cm] {\footnotesize $D[h_S]$ (w.p. $1-\delta$)};
    \draw [decorate, decoration={brace, amplitude=10pt}] (-0.3,0.05) -- (0.3,0.05) node [black, midway, yshift=0.55cm] {\footnotesize $\mathbb{E}[D[h_S]]$};
  \end{tikzpicture}
\end{figure}
One can observe that despite the fact the bound for $\mathbb{E}[D[h_S]]$ is tighter, we have no guarantees that the actual of $D[h_S]$ lies in the interval $\left[-\beta, \beta \right]$. On the other hand, it possible to assure with probability $1-\delta$ will be in $\left[-\beta-\epsilon, \beta+\epsilon \right]$.

\section*{Summary}
\textbf{Sufficient condition: Given enough samples we can achieve a good enough generalization.  However, typically in deep learning, we never have large enough data sets to get non-vacuous or meaningful bounds.}

\begin{table}[h!]
\centering
\begin{tabular}{|c  | c|} 
 \hline
 \rowcolor{\shadethmcolor}
 \textbf{Last part} & \textbf{Next part} \\ [0.5ex] 
 \hline
 PAC Bounds &  Stability \\
 Occam Bounds & PAC Bayes \\
 PAC Bayes Bounds & (Practical) Generalization \\
 Stability Bounds &  \\
\hline
\end{tabular}
\end{table}

How can we go from PAC Bayes to a non-vacuous generalization bound?

\vspace{0.3cm}
By sacrificing some data as part of a dedicated test set, we can measure test set generalization and achieve a tighter bound than the weak population bounds.  See \textit{Tutorial on Practical Prediction Theory for Classification} \cite{Langford:2005:TPP:1046920.1058111} for a comprehensive examination.

\section*{Empirical Risk Minimization + Regularization is Stable}

In this section, we will show that ERM with regularization is stable. We will see how the regularization term gives us a strongly convex objective, which then allows us to relate differences in objective value to differences in the weights of the hypotheses. We start by introducing some new notation that helps us refer to hypotheses by their weights: 

$$\hat{R}_S(w) \triangleq \hat{R}_S(h_w)$$ where $h_w$ is a model parameterized by weights $w$. \\

Correspondingly, we update the notation inside the loss function as follows:
\begin{align*}
    l(h,z)&\equiv l(h(x), y) \\
    l(h_w, z) &\equiv l(w,z)
\end{align*}

\begin{thm}[ERM with regularization is $\beta$-stable]
Under the assumption that $\hat{R}_S(w)$ is convex and $l(\cdot|z)$ as a function of $w$ is L-Lipschitz $\forall z$, Empirical Risk Minimization and Regularization is $\beta$-uniformly stable where

\[
\beta = \frac{4L^2}{\lambda n}
\]
\end{thm}
\begin{proof}
The objective function to be optimized can be written as
$$f_S(w)=\hat{R}_S(w) + \frac{\lambda}{2}||w||_2^2.$$

Consider weights $u,v$ for two different models. We have by definition
$$f_S(v) - f_S(u) = [\hat{R}_S(v) + \frac{\lambda}{2}||v||_2^2] - [\hat{R}_S(u) + \frac{\lambda}{2}||u||_2^2].$$

We perturb the dataset by replacing the data point at $i$ with $z_i'$ sampled i.i.d. from $\mathcal{D}$. Now we get:
\begin{align*}
    f_S(v) - f_S(u) &= \hat{R}_{S_{i,z_i'}}(v) + \frac{\lambda}{2}||v||_2^2 - (\hat{R}_{S_{i,z_i'}}(u) + \frac{\lambda}{2}||u||^2_2) + \frac{l(v, z_i) - l(v, z_i')}{n} - \frac{l(u, z_i) - l(u,z_i')}{n} \\
    &= f_{S_{i,z_i'}}(v) - f_{S_{i,z_i'}}(u) + \frac{l(v, z_i) - l(v, z_i')}{n} - \frac{l(u, z_i) - l(u,z_i')}{n}.
\end{align*}
Now we can substitute $v=\mathcal{A}(S_{i,z_i'})$ and $u=\mathcal{A}(S)$, where $\mathcal{A}$ is the ERM with regularization learning algorithm.
\begin{align*}
    f_S(\mathcal{A}(S_{i,z_i'})) - f_S(\mathcal{A}(S)) 
    =& f_{S_{i,z_i'}}(\mathcal{A}(S_{i,z_i'})) - f_{S_{i,z_i'}}(\mathcal{A}(S)) \\ &+ \frac{l(\mathcal{A}(S_{i,z_i'}), z_i) - l(\mathcal{A}(S_{i,z_i'}), z_i')}{n} - \frac{l(\mathcal{A}(S), z_i) - l(\mathcal{A}(S),z_i')}{n}.
\end{align*}

Since $\mathcal{A}$ is the ERM + regularization algorithm, $\mathcal{A}(S_{i, z_i'})$ minimizes $f_{S_{i, z_i'}}(\mathcal{A}(S_{i, z_i'}))$. Thus, we have 
\begin{align*}
f_{S_{i,z_i'}}(\mathcal{A}(S_{i, z_i'})) = \min_w f_{S_{i,z_i'}}(w) \\
\implies \forall w f_{S_{i,z_i'}}(w) \geq f_{S_{i,z_i'}}(\mathcal{A}(S_{i, z_i'}))\\
\implies f_{S_{i,z_i'}}(\mathcal{A}(S_{i,z_i'})) - f_{S_{i,z_i'}}(\mathcal{A}(S)) \leq 0
\end{align*}

Using this inequality and our assumption that $l(\cdot|z)$ is L-Lipschitz, we have
\begin{align}
f_S(\mathcal{A}(S_{i,z_i'})) - f_S(\mathcal{A}(S)) &\leq \frac{l(\mathcal{A}(S_{i,z_i'}), z_i) - l(\mathcal{A}(S), z_i)}{n} - \frac{l(\mathcal{A}(S_{i,z_i'}), z_i') - l(\mathcal{A}(S),z_i')}{n} \nonumber\\
&\leq 2\frac{L}{n}||\mathcal{A}(S)-\mathcal{A}(S_{i,z_i'})||_2 \label{eq:one}
\end{align}

Next, since we assumed that $\hat{R}_S(w)$ is convex, we know that $f_S(w)$ is $\lambda$-strongly convex. Using the property of $\lambda$-strongly convex functions, we can write:

\begin{align}
    f_S(\mathcal{A}(S_{i,z_i'})) - f_S(\mathcal{A}(S)) &\geq \nabla f_S(\mathcal{A}(S))^\top (\mathcal{A}(S) - \mathcal{A}(S_{i,z_i'})) +  \frac{\lambda}{2}||\mathcal{A}(S_{i,z_i'}) - \mathcal{A}(S)||_2^2 \\
    &= \frac{\lambda}{2}||\mathcal{A}(S_{i,z_i'}) - \mathcal{A}(S)||_2^2
\label{eq:two}
\end{align}


Since $\mathcal{A}(S)$ is the minimizer of $f_S$, the gradient evaluated at $\mathcal{A}(S)$ is $0$ and the first-order term disappears. \\

From \ref{eq:one} and \ref{eq:two} we get:
\begin{align}
    ||\mathcal{A}(S) - \mathcal{A}(S_{i,z_i'})|| \leq \frac{4L}{\lambda n}
\label{eq:three}
\end{align}

This means that the resulting hypotheses from the original dataset and a perturbed dataset have weights that become arbitrarily close for large $n$.\\
% If we perturb the data by a single element, we learn $\mathcal{A}$ that can become arbitrarily close for large $n$.

Finally, we use \ref{eq:three} and the $L$-Lipschitz property of $l(\cdot,z)$ to go from a bound on the parameters to a bound on the loss values, which completes the $\beta$-uniform stability proof:
$$\sup_z|l(\mathcal{A}(S), z) - l(\mathcal{A}(S_{i,z_i'}), z)| \leq \frac{4L^2}{\lambda n} = \beta$$
\end{proof}


\section*{Stochastic Gradient Descent (SGD) is Stable}

\subsection*{Stability Theorem}
Recall the SGD update formula,

\begin{equation}
    w_{t+1} = w_t - \alpha_t \nabla_w l(w_t, z_{i,t}), i_t   \sim \text{uniform}(1, \cdots, n )
\end{equation}

where $w_t$ is the weight iterate at time $t$, $\alpha_t$ is an (annealing) learning rate at time $t$ and $l(w_t, z_{i,t})$ is the computed loss for the current weight iterate for a particular example $z_{i,t}$.
 

\begin{thm}
If $f(\cdot, z)$ is $\gamma$-smooth, convex and $L$-Lipschitz, then Stochastic Gradient Descent is $\beta$-uniformly stable where

\[
\beta \leq \frac{2L^2}{n} \sum_{t=1}^{T} \alpha_t
\]
\end{thm}

\textbf{Note:} \\
Here, we are no longer requiring the function to be strongly convex.  Additionally, this result holds for a finite number of steps $T$.  

\subsection*{Stability Proof (Rough Outline)}
We will consider two runs of the SGD algorithm.  One run will be on the original dataset $S$ and the other run will be on the dataset $S_{i,z_i'}$.  Recall that $S_{i,z_i'}$ denotes the same dataset as $S$ but with the $i^{th}$ element swapped with element $z_i'$.  In order to compare the stability between the two runs, we maintain the same order of element selection (same random seed) for $t=1, \cdots, T$. 

\vspace{0.2cm}

\begin{defn} \label{def:delta}
Let $w_t$ denote the iterate of the SGD algorithm on the dataset $S$ at time step $t$, and let $w_t'$ denote the iterate on the dataset $S_{i,z_i'}$ at time step $t$.
    \[
    \delta_t = || w_t - w_t'||
    \]
\end{defn}

We can write the expectation of the difference $\delta_{t+1}$ as follows using the Law of Total Expectation:

\begin{equation} \label{eqn: sgd_prob_decomp}
    E[\delta_{t+1}] = P(i_t = i) E[\delta_{t+1} | i_t = i] + P(i_t \neq i) E[\delta_{t+1} | i_t \neq i]
\end{equation}

Next, we introduce two Lemmas:

\begin{lemma} \label{lem: lem_1}
The expectation of the difference does not grow on unperturbed steps.
    \[
    E[\delta_{t+1} | i_t \neq i] \leq E[\delta_t]
    \]
\end{lemma}
\begin{proof}
First, note that convexity and $\gamma$-smoothness implies that the gradients are co-coercive for a  function $f$: 
\[ 
\langle\,\nabla f(v)- \nabla f(w), v-w \rangle \geq \frac{1}{\gamma}||\nabla f(v)- \nabla f(w)||^2
\]
By substituting the SGD update step and expanding the square, the difference in weights can be expressed as:
\begin{align*}
    ||w_{t+1}-w'_{t+1}||^2 &= ||w_t-w'_t||^2 - 2\alpha_t \langle \nabla f(w_t)- \nabla f(w'_t), w_t-w'_t \rangle +\alpha^2 ||\nabla f(w_t)- \nabla f(w'_t)||^2\\
    &\leq ||w_t-w'_t||^2 - (2 \alpha_t / \gamma - \alpha_t^2) ||\nabla f(w_t)- \nabla f(w'_t)||^2\\
    &\leq ||w_t-w'_t||^2
\end{align*}

In the last step, we assume that we use a sufficiently small step size $\alpha_t$ so that the gradient updates are non-expansive. In particular, we assume that $\alpha_t \leq \frac{2}{\gamma}$.\\

Using definition \ref{def:delta}, we get:
    \[
    ||w_{t+1}-w'_{t+1}|| = \delta_{t+1} \leq ||w_{t}-w'_{t}|| = \delta_{t}
    \]
\end{proof}
\begin{lemma} \label{lem: lem_2} The growth in the distance is bounded after a perturbed step.
    \[
    E[\delta_{t+1} | i_t = i] \leq E[\delta_t] + 2 \alpha_t L
    \]
where $L$ is the Lipschitz value.
\end{lemma}
\begin{proof}
We know that 
    \[\delta_{t+1}=||w_{t+1}-w_{t+1}'||=||w_t-\alpha_t \nabla l(w_t, z_{i_t}) - ( w_t'-\alpha_t \nabla l(w_t', z_{i_t}))||.
    \]
Using the triangle inequality we can write
    \[\delta_{t+1} \leq ||w_t - w_t'|| + \alpha_t||\nabla l(w_t,z_{i_t}) - \nabla l(w_t',z_{i_t})||.
    \]
Since $l(\cdot, z)$ is L-lipschitz,
    \[\delta_{t+1} \leq \delta_t + 2\alpha_t L.
    \]
Taking expectation on both sides, we get
    \[E[\delta_{t+1}|i_t=i] \leq E[\delta_t] + 2\alpha_t L.
    \]
\end{proof}

Lemmas \ref{lem: lem_1}, \ref{lem: lem_2} tell us that the distance does not grow when we are following the sequence of data points, and every time we hit the perturbed data point, the distance can grow again but this increase is bounded. Hitting a perturbed point will happen every $n$ steps. Using these results, we can rewrite Equation \ref{eqn: sgd_prob_decomp} as:

\begin{align}
E[\delta_{t+1}] &= P(i_t=i)E[\delta_{t+1}|i_t=i] + P(i_t \neq i)E[\delta_{t+1}|i_t \neq i]] \\
&\leq \frac{1}{n}(E[\delta_t] + 2\alpha_t L) + \left(1 - \frac{1}{n}\right)E[\delta_t]\\
& \leq E[\delta_t] + \frac{2 \alpha_t L}{n},
\end{align}
\\
which when recursively unrolled yields the following final $\delta_T$:
\begin{equation}
    E[\delta_T] = E[|| w_T - w_T' ||] \leq  \sum_{t=0} ^ {T-1}  \frac{ 2 \alpha_t L }{n}.
\end{equation}

Finally, the $L$-Lipschitz property of $f(\cdot,z)$ tells us that
$$E[ |f(w_T, z) - f(w_T', z)|] \leq L E[|| w_T - w_T' ||].$$ We employ the notion of $\beta$-uniform stability for randomized algorithms as 
$$\sup_z[E[|f(w_T, z) - f(w_T', z)|] \leq \beta.$$

Putting everything together, we complete the $\beta$-uniform stability proof as follows:
\begin{align}
    \sup_z[E[|f(w_T, z) - f(w_T', z)|] &\leq L E[|| w_T - w_T' ||]\\
    &= \sum_{t=0} ^ {T-1}  \frac{ 2 \alpha_t L^2 }{n}\\
    &= \beta
\end{align}

Since $\beta$ is $O(\frac{1}{n})$, we can conclude that SGD is \textbf{stable}.
%\subsection*{5.2 Noise and Fast Rates}
%
%\begin{prop}[Key Property]
%\[
%	\mathrm{Var}(f) \leq Pf
%\]
%\end{prop}
%Allows faster rates of convergence for $L(g_n^*)- \inf_{g \in \mathcal{C}} L(g)$.
%
%For example, when $\infgc L(g) =0$, then
%\[
%	L(g^*_n) - \infgc L(g) \ll \supgc (L(g) - L_n(g)).
%\]

%----------------------------------------
% \section*{Acknowledgments} 

%----------------------------------------
\bibliographystyle{abbrvnat}
\bibliography{Refs/lec9}
%----------------------------------------
\end{document}